\documentclass[b4paper,landscape,twocolumn,10pt,fleqn]{jsarticle}
\usepackage{ascmac}
\usepackage{fancybox}
\usepackage{amsmath}
\usepackage{amssymb}
\usepackage{tabularray}
\usepackage[dvipdfmx,hiresbb]{graphicx}
\usepackage[version=4]{mhchem}
\usepackage{wrapfig}
\usepackage{mathcomp}
\usepackage{tikz}
\usepackage{float}
\usetikzlibrary{intersections,calc}
\usetikzlibrary{quotes,angles}
\usepackage{bm}
\usepackage{amsthm}
\usepackage[dvipdfmx,hiresbb]{graphicx}
\usepackage[version=4]{mhchem}
\usepackage{wrapfig}
\setlength{\oddsidemargin}{-5mm}
\setlength{\topmargin}{-20mm}
\setlength{\textwidth}{325mm}
\setlength{\textheight}{230mm}
\pagestyle{empty}



\begin{document}
\vspace{5cm}
\centerline{\Large\textbf{第2学年 物理(3期) 試験問題}}
\vspace{0.5cm}
\centerline{古川 創一(作成加村)}
\vspace{0.3cm}
\centerline{\Large{2025年2月10日}}
\vspace{0.3cm}

\fbox{1}
一辺$a$の正方形の一巻きコイルが,質量$M$の台車の上に固定されている.
Fig.1のように,この台車と糸でつながれた質量$m$の重りが,滑車を介してつるされている.台車は$x$軸と平行に摩擦なく動くことができる.
また,コイルの面に垂直方向に磁場$B_y$が存在しており,$B_y$は位置$x$のみの関数として$B_y(x)=Ax(A>0)$と表される.コイルの質量と自己インダクタンスは無視し,

\end{document}